%请用win+R打开TeXworks
%再用XeLaTeX编译
%感谢简明的入门教程https://liam0205.me/2014/09/08/latex-introduction/

%用来写中文
\documentclass[UTF8]{ctexart}

%控制页面大小与页边距
\usepackage{geometry}
\geometry{papersize={21cm,29.7cm}}
\geometry{left=3cm,right=2cm,top=2.5cm,bottom=2.5cm}

%控制图表不乱跑
\usepackage{float}

%控制页眉页脚
\usepackage{fancyhdr}
\pagestyle{fancy}
\lhead{}
\chead{}
\rhead{}
\lfoot{}
\cfoot{\thepage}%页码在页脚居中
\rfoot{}
\renewcommand{\headrulewidth}{0pt}%页眉和正文之间分割线宽度为0(就是没有)
\renewcommand{\headwidth}{\textwidth}
\renewcommand{\footrulewidth}{0pt}%页脚和正文之间分割线宽度为0(也是没有)

%用来插图
\usepackage{graphicx}

%文章信息
\title{我是题目}
\author{我是作者的名字}
\date{\today}%今天的日期

%开始
\begin{document}
\maketitle

%一号标题
\section{我是一号标题}
相当于markdown里面的\# 我是一号标题
%二号标题
\subsection{我是二号标题}
相当于markdown里的\#\# 我是二号标题
%三号标题
\subsubsection{我是三号标题}
相当于markdown里的\#\#\# 我是三号标题

%插入图表
\begin{figure}[H]%控制图表在原地不动
\centering%居中
\includegraphics[width = .8\textwidth]{i am a picture.png}%控制图片宽度为页面宽度的百分之八十,插入图片
\caption{我是一张图片}%图片名
\label{fig:1}%图片编号,这里是1
\end{figure}

%结束
\end{document}
%请用win+R打开TeXworks
%再用XeLaTeX编译
%感谢简明的入门教程https://liam0205.me/2014/09/08/latex-introduction/

%用来写中文
\documentclass[UTF8]{ctexart}

%控制页面大小与页边距
\usepackage{geometry}
\geometry{papersize={21cm,29.7cm}}
\geometry{left=3cm,right=2cm,top=2.5cm,bottom=2.5cm}

%控制图表不乱跑
\usepackage{float}

%数学
\usepackage{amsmath}
\usepackage{amssymb}

%高亮代码块;
\usepackage{listings}
\lstset{language=matlab}%这里用的语言是matlab

%控制页眉页脚
\usepackage{fancyhdr}
\pagestyle{fancy}
\lhead{}
\chead{}
\rhead{}
\lfoot{}
\cfoot{\thepage}%页码在页脚居中
\rfoot{}
\renewcommand{\headrulewidth}{0pt}%页眉和正文之间分割线宽度为0(就是没有)
\renewcommand{\headwidth}{\textwidth}
\renewcommand{\footrulewidth}{0pt}%页脚和正文之间分割线宽度为0(也是没有)

%用来插图
\usepackage{graphicx}

%列表缩进
\usepackage{iitem}

%用来画表格
\usepackage{booktabs}
\usepackage{array}
\usepackage{multirow}%合并单元格

% 参考文献写法1
\bibliographystyle{plain}

%插入导言
%文章信息
\title{我是题目}
\author{我是作者的名字}
\date{\today}%今天的日期

%开始
\begin{document}
%简单标题
%\maketitle

%标题页
\begin{titlepage}
\newgeometry{left=2cm,right=2cm,top=4.5cm,bottom=2.5cm}%调整页边距
\begin{center}
	{\heiti\Huge 我是标题}\\[30mm]
	{\large 作者 \footnote{我是作者\ 有何贵干}}\\[5mm]
	作者单位\\[15mm]
	\begin{minipage}[c]{200pt}
		\large 姓\qquad 名:作者\\
		单位\\
	\end{minipage}\\[100mm]
	\textbf{\heiti 中国\quad 北京\\ 
	BEIJING, CHINA\\
	\today}
\end{center}
\end{titlepage}

\restoregeometry%恢复原有页边距
%插入标题页

%目录
\tableofcontents
\newpage

%图目录
\listoffigures

\newpage%分页


%摘要
\begin{abstract}
\addcontentsline{toc}{section}{摘要}
这是摘要。
\end{abstract}

\section{Chapter2}
\subsection{装作这里还有一节}
Let's play with \textbf{\emph{\underline{\LaTeX}}}!


\section{Chapter2}
\subsection{装作这里还有一节}
Let's play with \textbf{\emph{\underline{\LaTeX}}}!


Let's play with \textbf{\emph{\underline{\LaTeX}}}!\cite{1}

引用一下图\ref{figure:1}
%一号标题
\section{我是一号标题}
相当于markdown里面的\# 我是一号标题
%二号标题
\subsection{我是二号标题}
相当于markdown里的\#\# 我是二号标题
%三号标题
\subsubsection{我是三号标题}
相当于markdown里的\#\#\# 我是三号标题
\paragraph{我是第一段}
相当于第一段
%插入图表
\begin{figure}[H]%控制图表在原地不动
\centering%居中
\includegraphics[width = .8\textwidth]{pic.png}%控制图片宽度为页面宽度的百分之八十,插入图片
\caption{我是一张图片}%图片名
\label{figure:1}%图片编号,这里是1
\end{figure}

%插入编号列表
\begin{enumerate}
\item item1
\item item2
\end{enumerate}

%波浪线
数学$\sim$
文字\textasciitilde

%插入无序列表
\begin{itemize}
\item item1
	\iitem item1.1
		\iiitem item 1.1.1
			\ivtem item1.1.1.1
\item item2
\end{itemize}

%插入横线

\rule[-10pt]{14.3cm}{0.05em}

%插入引用
\begin{quotation}
i am a block quote
\end{quotation}

%插入代码
\begin{lstlisting}
% 这里是matlab代码
\end{lstlisting}

%插入表格
\begin{table}[!htbp]%!: 尽可能按参数指定方式处理表格浮动位置,h: 当前位置,t: 下一页页首,b: 当前页页尾
\newcommand{\tabincell}[2]{\begin{tabular}{@{}#1@{}}#2\end{tabular}}
\centering
\begin{tabular}{m{2cm}b{2cm}p{2cm}}%表格,不画竖线
\toprule%画顶端线
指标1&\tabincell{c}{指标2\\指标2}&指标3\\
\midrule%画中间横线
参数1&参数2&参数3\\
\hline%画横线
参数1&参数2&参数3\\
\bottomrule%画底端横线
\end{tabular}
\caption{这是一张表}\label{table}
\end{table}

%控制表格自动换行
\begin{tabular}{ccp{3cm}p{1.5cm}p{3cm}c}
\toprule
股票代码&	证券简称&	机构名称&	一级行业名称&	二级行业名称&	年份\\
\midrule
600338.SH&	西藏珠峰&	西藏珠峰资源股份有限公司&	采矿业&	有色金属矿采选业&	2013-2017\\
600354.SH&	敦煌种业&	甘肃省敦煌种业(集团)股份有限公司&	农、林、牧、渔业&	农业&	2006-2010\\
000998.SZ&	隆平高科&	袁隆平农业高科技股份有限公司&	农、林、牧、渔业&	农业&	2006-2010\\
600320.SH&	振华重工&	上海振华重工(集团)股份有限公司&	制造业&	专用设备制造业&	2003-2007\\
300072.SZ&	三聚环保&	北京三聚环保新材料股份有限公司&	制造业&	化学原料和化学制品制造业&	2012-2016\\
000792.SZ&	盐湖股份&	青海盐湖工业股份有限公司&	制造业&	化学原料和化学制品制造业&	2003-2007\\
000063.SZ&	中兴通讯&	中兴通讯股份有限公司&	制造业&	计算机、通信和其他电子设备制造业&	2003-2007\\
\bottomrule
\end{tabular}

\begin{table}[!htbp]%!: 尽可能按参数指定方式处理表格浮动位置,h: 当前位置,t: 下一页页首,b: 当前页页尾
\centering
\begin{tabular}{m{2cm}b{2cm}p{2cm}}%表格,不画竖线
\toprule%画顶端线
指标1&指标2&指标3\\
\midrule%画中间横线
\multirow{2}*{参数}&参数2&参数3\\
\cline{2-3}
~&参数2&参数3\\
\bottomrule%画底端横线
\end{tabular}
\caption{这是一张表}\label{table}
\end{table}




%插入公式
行内公式$1+1=2$\\

行间公式
$$
1+1=2
$$

\[
1+1=2
\]

对齐不标号
\begin{align*}
1+1&=2\\
1+2&=3
\end{align*}



%参考文献写法1
\addcontentsline{toc}{section}{参考文献}
\bibliography{4bib.bib}


%参考文献写法2
%\begin{thebibliography}{0}
%\addcontentsline{toc}{section}{参考文献}
%\bibitem{参考文献1.{2018}}
%参考作者, 参考文献, 2018-12-21
%\end{thebibliography}



%结束
\end{document}
